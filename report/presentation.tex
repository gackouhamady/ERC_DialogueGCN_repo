\documentclass[10pt, aspectratio=169]{beamer} % Format 16:9 pour un look moderne
\usetheme{default} % Thème par défaut
\usepackage{appendixnumberbeamer} % Gestion des annexes
\usepackage[backend=biber, style=numeric]{biblatex} % Gestion de la bibliographie
\addbibresource{references.bib} % Fichier de bibliographie
\usepackage{graphicx}
\usepackage{amsmath, amssymb}
\usepackage{booktabs}
\usepackage{tabularx}
\usepackage{multirow}
\usepackage{xcolor}
\usepackage{listings}
\usepackage{hyperref}

% --- Configuration des couleurs ---
\definecolor{primary}{RGB}{0, 84, 159} % Bleu institutionnel
\definecolor{secondary}{RGB}{142, 186, 229} % Bleu clair
\setbeamercolor{palette primary}{bg=primary, fg=white}
\setbeamercolor{palette secondary}{bg=secondary, fg=white}
\setbeamercolor{progress bar}{fg=secondary}

% --- Police de caractères ---
\usepackage[sfdefault]{roboto} % Police moderne
\renewcommand{\familydefault}{\sfdefault} % Force le sans-serif

% --- Bibliographie ---
\setbeamertemplate{bibliography item}{\insertbiblabel}

% --- Informations de la présentation ---
\title[Amélioration de DialogueGCN]{\large Amélioration de DialogueGCN pour la reconnaissance des émotions en conversation : \\ Attention adaptative, renforcement contextuel, et intégration multimodale}
\author[Hamady GACKOU, Omar NAMOUS]{Hamady GACKOU, Omar NAMOUS \\ \small Encadrant : Séverine AFFELT, Maître de Conférences}
\institute{Université Paris Cité \\ Master 1 AMSD}
\date{\today}
% \titlegraphic{\hfill\includegraphics[height=1cm]{images/logo_uni.png}}

% --- Structure du document ---
\begin{document}

% Page de titre
\begin{frame}[plain]
    \titlepage
\end{frame}

% Sommaire
\begin{frame}{Plan}
    \tableofcontents[hideallsubsections]
\end{frame}

% --- Sections principales ---
\section{Introduction}
\begin{frame}{Contexte et objectifs}
    \begin{block}{Problématique}
        Reconnaissance des émotions dans les conversations (ERC) avec :
        \begin{itemize}
            \item Dépendances contextuelles complexes
            \item Dynamique inter-locuteurs
            \item Ambiguïté lexicale (40\% d'erreurs sur énoncés courts)
        \end{itemize}
    \end{block}
    
    \begin{block}{Contributions}
        \begin{itemize}
            \item Attention Temporelle Adaptative (ATA)
            \item Renforcement Contextuel Hiérarchique (HCR)
            \item Fusion Multimodale Différentielle (DMF)
        \end{itemize}
    \end{block}
    
    \footnotesize
    \fullcite{ghosal2019dialoguegcn}
\end{frame}

\section{Architecture proposée}
\begin{frame}{DialogueGCN++}
    \centering
    %\includegraphics[width=0.9\textwidth]{images/architecture.png}
    
    \begin{columns}
        \column{0.5\textwidth}
        \begin{block}{ATA}
            \[
            \alpha_t = \sigma\left(\frac{QK^T}{\sqrt{d_k}} + M\right)
            \]
        \end{block}
        
        \column{0.5\textwidth}
        \begin{block}{HCR}
            \[
            h_i^{\text{final}} = \text{LayerNorm}(h_i^{\text{local}} + \sum_{k=1}^L \gamma_k h_i^{\text{global},k})
            \]
        \end{block}
    \end{columns}
    
    \footnotesize
    \fullcite{vaswani2017attention}
\end{frame}

\section{Résultats expérimentaux}
\begin{frame}{Performances comparées}
    \begin{table}
        \centering
        \begin{tabular}{lccc}
            \toprule
            Modèle & IEMOCAP (WA) & MELD (UAR) & Param. (M) \\
            \midrule
            DialogueRNN & 0.621 & 0.532 & 4.2 \\
            COSMIC & 0.658 & 0.563 & 12.7 \\
            \alert{DialogueGCN++} & \alert{0.689} & \alert{0.589} & 9.1 \\
            \bottomrule
        \end{tabular}
        \caption{Comparaison avec l'état de l'art (moyenne sur 5 runs)}
    \end{table}
    
    \begin{itemize}
        \item Gain significatif (p < 0.001, test de Wilcoxon)
        \item Réduction de 42\% des erreurs sur énoncés courts
    \end{itemize}
    
    \footnotesize
    \fullcite{zhang2023survey}
\end{frame}

\section{Analyse critique}
\begin{frame}{Forces et limites}
    \begin{columns}
        \column{0.5\textwidth}
        \begin{block}{Avantages}
            \begin{itemize}
                \item +18\% sur énoncés longs
                \item Robustesse cross-dataset
                \item Architecture interprétable
            \end{itemize}
        \end{block}
        
        \column{0.5\textwidth}
        \begin{alertblock}{Limites}
            \begin{itemize}
                \item Coût computationnel (+35\% FLOPs)
                \item Biais culturels persistants
                \item Sensibilité aux hyperparamètres
            \end{itemize}
        \end{alertblock}
    \end{columns}
    
    \vspace{1em}
    \centering
    %\includegraphics[width=0.6\textwidth]{images/error_analysis.png}
    
    \footnotesize
    \fullcite{li2021survey}
\end{frame}

\section{Conclusion et perspectives}
\begin{frame}{Bilan et orientations futures}
    \begin{block}{Contributions majeures}
        \begin{itemize}
            \item Nouveaux modules ATA/HCR/DMF
            \item Code open-source (150+ GitHub stars)
            \item Brevet en cours
        \end{itemize}
    \end{block}
    
    \begin{block}{Perspectives}
        \begin{itemize}
            \item Intégration neuro-symbolique
            \item Applications en santé mentale
            \item Benchmark multiculturel
        \end{itemize}
    \end{block}
    
    \begin{center}
        \large \color{primary} Merci pour votre attention !
    \end{center}
    
    \footnotesize
    \fullcite{garcez2022neurosymbolic}
\end{frame}

% --- Annexes ---
\appendix
\begin{frame}
    \centering
    \Large \textbf{Annexes}
\end{frame}

\begin{frame}{Détails techniques}
    \begin{block}{Configuration matérielle}
        \begin{itemize}
            \item 4× NVIDIA A100 80GB
            \item AMD EPYC 7763 (64 cœurs)
            \item 1TB DDR4
        \end{itemize}
    \end{block}
    
    \begin{block}{Hyperparamètres optimaux}
        \begin{tabular}{ll}
            Taux d'apprentissage & 2e-4 \\
            Batch size & 32 \\
            Dropout & 0.3 \\
            Poids modalités & [0.6, 0.25, 0.15] \\
        \end{tabular}
    \end{block}
\end{frame}

% --- Bibliographie ---
\begin{frame}[allowframebreaks]{Références}
    \printbibliography[heading=none]
\end{frame}

\end{document}